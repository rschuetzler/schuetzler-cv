%------------------------------------
% Dario Taraborelli
% Typesetting your academic CV in LaTeX
%
% URL: http://nitens.org/taraborelli/cvtex
% DISCLAIMER: This template is provided for free and without any guarantee 
% that it will correctly compile on your system if you have a non-standard  
% configuration.
% Some rights reserved: http://creativecommons.org/licenses/by-sa/3.0/
% Requires DejaVuSansMono and LinLibettine fonts
% Download Deja Vu here: http://dejavu-fonts.org/wiki/Download
% and lin libertine here: http://www.dafont.com/linux-libertine.font
% For libertine you need R, RB, RBI, and RI fonts
%------------------------------------

%!TEX TS-program = xelatex
%!TEX encoding = UTF-8 Unicode

\documentclass[10pt, a4paper]{article}
\usepackage{fontspec} 

% DOCUMENT LAYOUT
\usepackage{geometry} 
\geometry{a4paper, textwidth=5.5in, textheight=8.5in, marginparsep=7pt, marginparwidth=.6in}
\setlength\parindent{0in}

% FONTS
\usepackage[usenames,dvipsnames]{xcolor}
\usepackage{xunicode}
\usepackage{xltxtra}
\defaultfontfeatures{Mapping=tex-text}
%\setromanfont [Ligatures={Common}, Numbers={OldStyle}, Variant=01]{Linux Libertine O}
%\setmonofont[Scale=0.8]{Monaco}
%%% modified by Karol Kozioł for ShareLaTeX use
\setmainfont[
  Ligatures={Common}, Numbers={OldStyle}, Variant=01,
  BoldFont=LinLibertine_RB.otf,
  ItalicFont=LinLibertine_RI.otf,
  BoldItalicFont=LinLibertine_RBI.otf
]{LinLibertine_R.otf}
\setmonofont[Scale=0.8]{DejaVuSansMono.ttf}

% ---- CUSTOM COMMANDS
\chardef\&="E050
\newcommand{\html}[1]{\href{#1}{\scriptsize\textsc{[html]}}}
\newcommand{\pdf}[1]{\href{#1}{\scriptsize\textsc{[pdf]}}}
\newcommand{\doi}[1]{\href{#1}{\scriptsize\textsc{[doi]}}}
% ---- MARGIN YEARS
\usepackage{marginnote}
\newcommand{\amper{}}{\chardef\amper="E0BD }
\newcommand{\years}[1]{\marginnote{\scriptsize #1}}
\renewcommand*{\raggedleftmarginnote}{}
\setlength{\marginparsep}{7pt}
\reversemarginpar

% HEADINGS
\usepackage{sectsty} 
\usepackage[normalem]{ulem} 
\sectionfont{\mdseries\upshape\Large}
\subsectionfont{\mdseries\scshape\normalsize} 
\subsubsectionfont{\mdseries\upshape\large} 

% PDF SETUP
% ---- FILL IN HERE THE DOC TITLE AND AUTHOR
\usepackage[%dvipdfm, 
bookmarks, colorlinks, breaklinks, 
% ---- FILL IN HERE THE TITLE AND AUTHOR
	pdftitle={Ryan Schuetzler - Vita},
	pdfauthor={Ryan Schuetzler},
	pdfproducer={http://nitens.org/taraborelli/cvtex}
]{hyperref}  
\hypersetup{linkcolor=blue,citecolor=blue,filecolor=black,urlcolor=MidnightBlue} 

% DOCUMENT
\begin{document}
{\LARGE Ryan Schuetzler}\\[1cm]
 Management Information Systems Department\\
 Center for the Management of Information\\
 McClelland Hall, Room 427\\
 1130 East Helen Street\\
 PO Box 210108\\
 Tucson, AZ \texttt{85721-0108}
U.S.A.\\[.2cm]
Phone: \texttt{(402) 915-3126}\\[.2cm]
email: \href{mailto:ryan@schuetzler.net}{ryan@schuetzler.net}\\
\textsc{url}: \href{http://www.schuetzler.net}{http://www.schuetzler.net}\\ 

%%\hrule
\section*{Current Position}
\emph{Doctoral Candidate}\\
Expected graduation: May 2015\\
Management Information Systems Department\\
Eller College of Management\\
University of Arizona


\section*{Research Interests}

My research interests center around users' adaptation and use of novel
communication technology. My primary research focuses on human interaction with
embodied conversational agents and chat bots, and the influence of those chat
bots on users' feelings and behavior toward those systems. I also study users'
adaptation to intelligent systems designed to detect deception. I am interested
to learn how people adapt their behavior, either consciously or unconsciously,
as they interact with novel systems. I focus on a design science methodology
following the Last Research Mile philosophy. Much of my research involves
laboratory experiments.

%%\hrule
\section*{Areas of Specialization}
%! Fill this out
Privacy $\bullet$ Human-Computer Interaction $\bullet$ Deception $\bullet$
Conversational Agents

%%\hrule
% \section*{Appointments held}
% \noindent
% \years{1903-1908}Swiss Patent Office, Bern\\
% \years{1908-1911}University of Bern\\
% \years{1911-1912}University of Zürich\\
% \years{1912-1914}Charles University of Prague\\
% \years{1914-1932}Prussian Academy of Sciences, Berlin\\
% \years{1920-1930}University of Leiden\\
% \years{1932-1955}Institute for Advanced Study, Princeton

%\hrule
\section*{Education}
\noindent
\years{2015}\textsc{Ph.D.} in Management Information Systems, University of Arizona\\
\years{2010}\textsc{MS} in Information Systems Management, Brigham Young University\\
\years{2010}\textsc{BS} in Information Systems, Brigham Young University

%\hrule
\section*{Grants, Honors \& Awards}
\noindent
\years{2011-2014} \$205,000 in research grants, Center for Identification
Technology Research\\\\
\years{2011} Science Foundation Arizona Graduate Research Fellow

\section*{Publications \& Talks}

\subsection*{Journal articles}
\noindent

\years{Under review} Twyman, N. T., Proudfoot, J. G., Schuetzler, R., Elkins, A. C., and
Derrick, D. C. (under review). Robustness of multiple indicators in controlled, automated
deception detection interviews. \emph{Journal of Management Information
  Systems}.\\

\years{In press}Lowry, P. B., Schuetzler, R. M., Giboney, J. S., \& Gregory, T. (in
press) Is trust always better than distrust? The potential value of distrust in
newer virtual teams engaged in short-term decision making. \emph{Group Decision and
  Negotiation}.\\

\years{2015}Burgoon, J. K., Schuetzler, R. \& Wilson, D. W. (2015) Kinesic patterning in
deceptive and truthful interactions.  \emph{Journal of Nonverbal Behavior}, 39(1), 1--24.\\

\years{2014}Burgoon, J. K., Proudfoot, J. G., Schuetzler, R. \& Wilson,
D. W. (2014) Patterns of nonverbal behavior associated with truth and
deception: Illustrations from three experiments. \emph{Journal of Nonverbal
  Behavior}, 38(3), 325--354.\\

\years{2011}Barlow, J. B., Giboney, J. S., Keith, M. J., Wilson, D. W.,
Schuetzler, R., Lowry, P. B., \& Vance, A. (2011). Overview and guidance on
agile development in large organizations. \emph{Communications of the
  Association for Information Systems}, 29(1), 25--44.

\subsection*{Conference proceedings}
\noindent
\years{2014}Schuetzler, R. M., Giboney, J. S., Grimes, G. M., \& Buckman,
J. (2014) Facilitating natural conversational agent interactions: Lessons from a
deception experiment. \emph{International Conference on Information Systems}. Auckland, New Zealand.\\

Dunbar, N. E., Jensen, M. L., Miller, C. H., Bessarabova, E.,
Straub, S., \ldots~\& Schuetzler, R. (2014) Mitigating cognitive bias through
the use of serious games: Effects of feedback. \emph{9th International
  Conference on
  Persuasive Technology}, Padova, Italy, May 21-23.\\

\years{2013}Twyman, N. W., Schuetzler, R., Proudfoot, J. G., \& Elkins,
A. (2013). A systems approach to countermeasures in credibility assessment
interviews. \emph{International Conference on Information Systems (ICIS)},
Milan, Italy, December 15--18.\\

Schuetzler, R. \& Wilson, D. W. (2013) Real-time embodied agent
adaptation. \emph{Hawaii International Conference on System Sciences}.\\

\years{2012}Schuetzler, R. (2012) Countermeasures and eye tracking deception
detection. \emph{Hawaii International Conference on System Sciences}.\\

Nunamaker, J. F., Jr., Burgoon, J. K., Twyman, N. W., Proudfoot, J. G.,
Schuetzler, R., \& Giboney, J. S. (2012) Establishing a foundation for human
credibility screening. \emph{2012 IEEE International Conference on Intelligence
  and Security Informatics (ISI)}.\\

Burgoon, J. K., Wilson, D. W., Schuetzler, R. \& Hass, M. (2012) Interactive
deception in group decision-making. \emph{National Communication Association
  Convention}.\\
\newpage
Burgoon, J. K., Schuetzler, R., \& Wilson, D. W. (2012) Uncovering hidden
patterning in interpersonal communication: Illustration with deceptive and
truthful interactions. \emph{National Communication Association Convention}.\\

\years{2011}Proudfoot, J. G., Giboney, J. S., Schuetzler, R. \& Durcikova,
A. (2011). Trends in phishing attacks: Suggestions for future
research. \emph{Americas Conference on Information Systems}.\\

\years{2010}Lowry, P. B., Giboney, J. S., Schuetzler, R., Richardson, J.,
Gregory, T., Romney, J., \& Anderson, B. (2010). The value of distrust in
computer-based decision-making groups. In \emph{Proceedings of the 43rd Hawaii
  International Conference on System Sciences}.

\subsection*{Invited Presentations and Panels}
\noindent

\years{2015}\emph{Lie to Me: Chatterbot Style}. Webinar for the Center for Identification
Technology Research, an NSF Industry/University Collaborative Research Center.\\

\years{2014}\emph{Identifying and Reducing Patient Drug Seeking}. Panel for first-year
medical students at University of Arizona College of Medicine.\\

\emph{A Mobile Interviewing Agent for Deception Detection}. Webinar for the
Center for Identification Technology Research, an NSF Industry/University Collaborative
Research Center.\\

\years{2012-2014}\emph{Deception and Automated Credibility Assessment}. Fort Huachuca
Military Intelligence Captain's Career course.



\section*{Teaching}

% \subsection*{Teaching Philosophy}
% My teaching is centered around one principle: I want students to love the
% material I teach in the class. I start each semester telling my students just
% that. I tell them that if they don't love them, I want them to at least like
% it; and if not like it, then at least understand it. In order to foster that
% attitude, I believe in creating an interactive classroom where I am not the only
% one instructing. I design my class in a way to allow students to contribute and
% teach each other. I also fully admit to my students that I do not know
% everything, and show my own love of the material by seeking answers to student
% questions that stump me during class.\\

% My in-class activities are centered around ensuring that students learn,
% understand, and apply their knowledge. Regular hands-on activities and in-class
% small group activities help students solidify their knowledge in a way that my
% lecturing cannot. In any way possible, I strive to give students the basic
% knowledge that they need in order to teach themselves. There is not enough time
% in a semester, or in an entire university education, to learn everything one
% must know in this world of ever-advancing technology. So I see my role as giving
% students the basic pieces that they can later put together in new and exciting
% ways. 

% I believe that the purpose of a university education is to teach students how to
% think and how to learn. With the ever-changing technological landscape, the need
% to memorize information is nearly nonexistent. As such, my instruction focuses
% on giving students the hands-on experience and skills that they could not easily
% Google in the future. I believe in equipping students with a level of knowledge
% and understanding of course topics so they can converse intelligently and
% understand reference material easily available on the web.\\

% My course instruction centers on class participation and understanding. When
% possible I bring hands-on lab exercises and learning activities to aid students
% in internalizing course material.

\subsection*{Courses Taught}

\years{2013-2015}MIS 307 -- Introduction to Business Data Communications
(Effectiveness: 4.6/5.0)

\years{2012}MIS 111 -- Introduction to Management Information Systems
(Effectiveness: 4.9/5.0)

\subsection*{Select Student Comments}

\begin{quote}
  I appreciated your passion and it encouraged me to work harder and study better for this
  course\ldots I fully digested the information you presented to us. After completing this
  course, I am seriously considering spending more time in the areas you taught us because
  I have found a curiosity for the area I did not imagine before.
\end{quote}

\begin{quote}
  I really enjoy class. The way the material has been amplified with your own knowledge
  and the way the slides are structured really help me to pay attention, take notes, and
  tell myself "Ooo, that's something I want to put down here. That's important."
\end{quote}

\begin{quote}
  Great teacher, made everything very easy to understand. Made class interesting and
  fun. Kept everyone engaged in learning.
\end{quote}

\begin{quote}
  I really enjoyed that Ryan truly cared if his students understood what was going on and
  comprehended the material.
\end{quote}

%\hrule
\section*{Service to the Profession}
Peer review work for:

\begin{itemize}
\itemsep0em
\item International Conference on Information Systems
\item Hawaii International Conference on System Sciences
\item Americas Conference on Information Systems
\item AIS Special Interest Group on Human-Computer Interaction (SIGHCI)
\item IEEE, SMC-A
\item Mensa Research Journal
\end{itemize}

% \section*{Consulting Clients}
% \begin{itemize}
% \itemsep0em
% \item Wal-Mart (Bentonville, AR)
% \begin{itemize}
%   \itemsep0em
%   \item 2010: Research and strategic development for agile software development
%     within Wal-Mart's current SDLC
% \end{itemize}
% \end{itemize}

\section*{Technical and Other Skills}

\begin{itemize}
\itemsep0em
\item Programming languages: Python, R, Java. A little Ruby, PHP, and C++.
\item Web application development with Python and Django
\item TCP/IP networking (formerly CCNA certified)
\item SQL \& database management
\item Drupal content management system
\item Linux and Windows servers
\item Spanish: Moderate proficiency
\end{itemize}


%\vspace{1cm}
\vfill{}
%\hrulefill

\begin{center}
{\scriptsize  Last updated: \today\- •\- 
% ---- PLEASE LEAVE THIS BACKLINK FOR ATTRIBUTION AS PER CC-LICENSE
Typeset in \href{http://nitens.org/taraborelli/cvtex}{
%\fontspec{Times New Roman}
\XeTeX }\\
% ---- FILL IN THE FULL URL TO YOUR CV HERE
% \href{http://nitens.org/taraborelli/cvtex}{http://nitens.org/taraborelli/cvtex}
}
\end{center}

\end{document}

%%% Local Variables:
%%% mode: latex
%%% TeX-master: t
%%% End:
